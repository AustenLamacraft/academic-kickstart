\[
\nonumber
\newcommand{\cN}{\mathcal{N}}
\newcommand{\br}{\mathbf{r}}
\newcommand{\bp}{\mathbf{p}}
\newcommand{\bk}{\mathbf{k}}
\newcommand{\bq}{\mathbf{q}}
\newcommand{\bv}{\mathbf{v}}
\newcommand{\pop}{\psi^{\vphantom{\dagger}}}
\newcommand{\pdop}{\psi^\dagger}
\newcommand{\Pop}{\Psi^{\vphantom{\dagger}}}
\newcommand{\Pdop}{\Psi^\dagger}
\newcommand{\Phop}{\Phi^{\vphantom{\dagger}}}
\newcommand{\Phdop}{\Phi^\dagger}
\newcommand{\phop}{\phi^{\vphantom{\dagger}}}
\newcommand{\phdop}{\phi^\dagger}
\newcommand{\aop}{a^{\vphantom{\dagger}}}
\newcommand{\adop}{a^\dagger}
\newcommand{\bop}{b^{\vphantom{\dagger}}}
\newcommand{\bdop}{b^\dagger}
\newcommand{\cop}{c^{\vphantom{\dagger}}}
\newcommand{\cdop}{c^\dagger}
\newcommand{\Nop}{\mathsf{N}^{\vphantom{\dagger}}}
\newcommand{\bra}[1]{\langle{#1}\rvert}
\newcommand{\ket}[1]{\lvert{#1}\rangle}
\newcommand{\inner}[2]{\langle{#1}\rvert #2 \rangle}
\newcommand{\braket}[3]{\langle{#1}\rvert #2 \lvert #3 \rangle}
\DeclareMathOperator{\sgn}{sgn}
\DeclareMathOperator{\tr}{tr}
\newcommand{\abs}[1]{\lvert{#1}\rvert}
\newcommand{\brN}{\br_1, \ldots, \br_N}
\newcommand{\xN}{x_1, \ldots, x_N}
\newcommand{\zN}{z_1, \ldots, z_N}
\]

Bose or Fermi statistics gives rise to correlations in the positions of
particles in a many body system, even in product states. These
correlations affect the energy of many particle states, and provide the
basis of the simplest approximate theories.

Reading: \{\% cite Nazarov:2013 \%\}, \{\% cite Baym:1969 \%\}

\begin{center}\rule{0.5\linewidth}{0.5pt}\end{center}

\hypertarget{correlation-functions}{%
\subsection{Correlation Functions}\label{correlation-functions}}

In {[}Lecture 1{]}(\{\{\textless{} ref ``many-body-wavefunctions''
\textgreater\}\}) we introduced the \textbf{pair distribution function}

\[
\rho_2(x_1,x_2) = N(N-1) \int dx_3\ldots dx_N \,\left|\Psi(x_1,x_2,\ldots,x_N)\right|^2,
\label{more_pdf}
\]

which measures the likelihood of finding a pair of particles at \(x_1\)
and \(x_2\). We also saw that \(\rho_2(x_1,x_2)\) could be related to{}
the expectation of the product of the density operators \(\rho(x)\) at
different points.

\[
\rho_2(x_1,x_2) = \bra{\Psi}\rho(x_1)\rho(x_2)\ket{\Psi} - \rho_1(x_1)\delta(x_1-x_2),
\label{more_rhorhonorm}
\]

where \(\rho_1(x)= \braket{\Psi}{\rho(x)}{\Psi}\). \(\rho_2(x,y)\) is
the simplest example of a \textbf{correlation function}. Note also that
the quantity \(\bra{\Psi} \rho_q \rho_{-q} \ket{\Psi}\), used to
quantify the crystalline order in a solid in {[}Lecture
3{]}(\{\{\textless{} ref ``elastic-chain'' \textgreater\}\}), is just
the Fourier transform of \(\rho_2(x_1,x_2)\). Higher correlation
functions, involving products of more than two density operators,
provide more detailed information on the distribution of the particles.

\hypertarget{correlation-functions-in-second-quantization}{%
\subsubsection{Correlation Functions in Second
Quantization}\label{correlation-functions-in-second-quantization}}

We evaluated \(\rho_2(x,y)\) for the 1D Fermi gas:

\[
\rho_2(x,y) = n^2\left[1 - \left(\frac{\sin[k_\text{F}(x-y)]}{k_\text{F}(x-y)}\right)^2\right].
\label{more_rho2evalFermi}
\]

Let's see how to reproduce this result using second quantization.
\(\rho_2(x,y)\) can be written in second quantized notation as

\[
\rho_2(x,y) =\braket{\Psi}{\pdop(x)\pdop(y)\pop(y)\pop(x)}{\Psi}.
\label{more_rho22ndquant}
\]

\begin{quote}
Perhaps the simplest way to get this formula is to use the
correspondence that we found in {[}Lecture 5{]}(\{\{\textless{} ref
``second-quantization'' \textgreater\}\}) \[
\begin{align}
\ket{\Psi}&\longleftrightarrow \Psi(x_1,\ldots, x_N)\nonumber\\
\psi(X)\ket{\Psi}&\longleftrightarrow \sqrt{N}\Psi(X,x_1,\ldots, x_{N-1})\\
\end{align}
\] Show that \(\eqref{more_pdf}\) implies \(\eqref{more_rho22ndquant}\).
\end{quote}

\(\eqref{more_rho22ndquant}\) differs from
\(\bra{\Psi}\rho(x)\rho(y)\ket{\Psi}\) by the ordering of operators.
Using the canonical commutation relations for the fields gives the
relation \(\eqref{more_rhorhonorm}\) for both bosons and fermions.

Let's evaluate \(\eqref{more_rho22ndquant}\) for the ground state of the
Fermi gas. We will find the result for a general product state, using
the same strategy as we used in the last lecture for the single particle
density matrix. That is, we substitute the expansion of the fields
\(\pop(x)\) and \(\pdop(x)\) in terms of the orthonormal single particle
states making up the product state

\[
\begin{align}
    \pop(x)=\sum_{\beta}  \varphi^{}_{\beta}(x)\aop_{\beta},\nonumber\\
  \pdop(x)=\sum_{\beta}  \varphi^*_{\beta}(x)\adop_{\beta}.
\end{align}
\]

This gives

\[
    \label{2nd_quant_CEval}
    \rho_2(x,y)=\sum_{\alpha, \beta, \gamma, \delta}\varphi^{*}_{\alpha}(x)\varphi^{*}_{\beta}(y)\varphi^{}_{\gamma}(y)\varphi^{}_{\delta}(x)\braket{\Psi}{\adop_{\alpha}\adop_{\beta}\aop_{\gamma}\aop_{\delta}}{\Psi}.
\]

If we are considering the expectation in a state of the form
\(\ket{\mathbf{N}}\), we can see that an annihilation operator for a
given single particle state must be accompanied by a creation operator
for the same state. There are therefore two possibilities

\[
\begin{align}
    &\alpha =\delta,\, \beta=\gamma, \text{ or }\nonumber\\
  &\alpha=\gamma,\, \beta=\delta,
\end{align}
\]

which give rise to two groups of terms

\[
\begin{align}
\braket{\mathbf{N}}{\adop_{\alpha}\adop_{\gamma}\aop_{\gamma}\aop_{\alpha}}{\mathbf{N}}&=N_{\alpha}N_{\gamma}\nonumber\\
    \braket{\mathbf{N}}{\adop_{\alpha}\adop_{\gamma}\aop_{\alpha}\aop_{\gamma}}{\mathbf{N}}&=\pm N_{\alpha}N_{\gamma}\qquad\text{if }\alpha\neq\gamma,
\end{align}
\]

the \(\pm\) corresponding to bosons and fermions respectively. Thus we
have

\[
\begin{align}
\rho_2(x,y)=\sum_{\alpha, \beta}N_\alpha N_\beta\left[\abs{\varphi_{\alpha}(x)}^2\abs{\varphi_{\beta}(y)}^2 \pm \varphi^*_\alpha(x)\varphi^{}_\alpha(y)\varphi^*_\beta(y)\varphi^{}_\beta(x) \right].
\label{more_rho2result}
\end{align}
\]

You might notice that this expression weights the case
\(\alpha=\beta=\gamma=\delta\) by a factor \(2N_\alpha^2\) when it
should have \(N_\alpha(N_\alpha-1)\) in the case of bosons, and zero for
fermions. On the other hand, such cases amount to a sum over a
\emph{single index}, where the general case is a sum over a double
index. It seems reasonable, then, that this accounting error is not
important in the thermodynamic limit. For instance, in the case that
\(\varphi_\alpha(x)\) are plane wave states we have the usual
prescription

\[
\sum_\alpha(\cdots) \longrightarrow L\int (\cdots)\frac{dk}{2\pi},
\]

assuming the integrand is smooth. In this case the error in
\(\eqref{more_rho2result}\) is a factor of \(L^{-1}\) smaller than what
we retain. We do however have to be careful in Bose condensates, where
one state has a finite fraction of the particles and this argument does
not apply.

We can express the result \(\eqref{more_rho2result}\) in terms of the
density and density matrix as

\[
\rho_2(x,y) = \rho_1(x)\rho_1(y) \pm g(x,y)g(y,x),
\label{more_rho2compact}
\]

which, in the case of the ground state of the Fermi gas, reproduces
\(\eqref{more_rho2evalFermi}\). We see that the correlation function
vanishes as the separation \(x-y\to 0\), because
\(g(x,x)=\langle\mathop{\rho(y)}\rangle\). This is, of course, another
manifestation of the exclusion principle: it is not possible for two
fermions to sit on top of each other. The scale of the `hole' in the
correlation function is of course set by the mean interparticle
separation, or Fermi wavelength. Note also the decaying oscillations,
indicating liquid-like correlations in the positions of the particles.
These are sometimes known as \textbf{Friedel oscillations}.

\{\{\textless{} figure src=``1DFermiGasCorrelation.png''
title=``Correlation function \(\rho_2(x,0)\) for the Fermi gas.''
numbered=``true'' lightbox=``true'' \textgreater\}\}

For bosons the situation is very different. If \(g(x,y)\to 0\) as
\(\abs{x-y}\to\infty\), the value of the correlation function as
\(\abs{x-y}\to 0\) is \emph{twice} the value at \(\abs{x-y}\to\infty\).
This characteristic behavior is often termed \textbf{bunching}: a pair
of bosons is more likely to be found at two nearby points than at two
distant points.

Nothing about the result \(\eqref{more_rho2compact}\) is special to 1D
of course: one just has to recalculate the density and density matrix.
Remember, though, that it does only apply to product states.

\hypertarget{the-hanbury-brown-and-twiss-effect}{%
\subsubsection{The Hanbury Brown and Twiss
Effect}\label{the-hanbury-brown-and-twiss-effect}}

The result for the density correlations \(\eqref{more_rho2result}\) or
\(\eqref{more_rho2compact}\) can be viewed as a kind of interference
effect that shows up in the correlations of the intensity of a quantum
waves, even when there is no interference in the intensity itself. To
illustrate this interpretation, we consider a classic experiment from
with Bose condensates \{\% cite Andrews1997 \%\}.

Consider a gas of \(N\) noninteracting bosons occupying the lowest
energy level of some potential well: a \textbf{Bose condensate}. If the
ground state wavefunction is \(\varphi_{0}(\br)\), the \(N\)-body
wavefunction for such a state is

\[
    \Psi(\br_1,\br_2,\ldots,\br_N)=\prod_i^N \varphi_0(\br_i),
  \label{2nd_quant_BoseGroundState}
\]

which we can write in second quantized notation as

\[
    \ket{\Psi}=\frac{1}{\sqrt{N!}}\left(\adop_0\right)^N\ket{\text{VAC}},
\]

where \(\adop_0\) creates a particle in the state \(\varphi_0(\br)\).
Imagine that we took another well, also filled with \(N\) bosons, and
placed it alongside the first. If we switch off the potentials at some
instant, the particles will fly out, with wavefunctions orginating in
the two wells overlapping. Precisely this experiment was reported in
\{\% cite Andrews1997 \%\}. What do we expect to see?

Let us denote by \(\varphi_L(\br)\) and \(\varphi_R(\br)\) the ground
states of two spatially separated potential wells. First, consider a
state where each boson is in a superposition of \(\varphi_L(\br)\) and
\(\varphi_R(\br)\). Such a situation could arise by starting from a
single well and adiabatically splitting it in two. We can write such a
state as

\[
    \ket{\bar N_L,\bar N_R}_\theta\equiv\frac{1}{\sqrt{N!}}\left[\sqrt{\frac{\bar N_L}{N}}e^{-i\theta/2}
    \adop_L+\sqrt{\frac{\bar N_R}{N}}e^{i\theta/2}\adop_R\right]^N\ket{\text{VAC}},
  \label{more_two}
\]

where \(\bar N_{L,R}\) are the expectation values of particle number in
each state \(N=\bar N_L+\bar N_R\). We allow the system to evolve for
some time \(t\), so that the two `clouds' begin to overlap (typically
achieved by allowing free expansion i.e.~turning off the confining
potentials). If we consider the time evolution in the Heisenberg picture
then, as we saw last time, the field operator obeys the free particle
Schrödinger equation (ignoring interactions)

\[
i\frac{\partial \pop(\br,t)}{\partial t} = -\frac{1}{2m}\nabla^2\pop(\br,t).
\]

We write the field operator as

\[
\pop(\br)=\varphi_L(\br,t)\aop_L+\varphi_R(\br,t)\aop_R+\cdots,
\]

where the wavefunctions \(\varphi_{L/R}(\br,t)\) are evolving freely,
and the dots denote the other states in some complete orthogonal set
that includes \(\varphi_L(\br)\) and \(\varphi_R(\br)\): we can ignore
them because they are empty. A calculation of the expectation value of
\(\rho(\br)=\pdop(\br)\pop(\br)\) gives

\[
    \begin{align}
    \rho_1(\br,t)=\bar N_L|\varphi_L(\br,t)|^2+\bar N_R|\varphi_R(\br,t)|^2+\overbrace{2\sqrt{\bar N_L \bar
    N_R}\mathrm{Re}\,e^{i\theta}\,\varphi^*_L(\br,t)\varphi_R(\br,t)}^{\equiv\rho_{\mathrm{int}(
    \br,t)}}.
  \label{dens_int}
    \end{align}
\]

If the clouds begin to overlap, the last term in \(\eqref{dens_int}\)
comes into play. Its origin is in quantum interference between the two
coherent subsystems, showing that the \emph{relative phase} has a real
physical effect.

\begin{quote}
Consider a Gaussian wavefunction of width \(R_0\) at time \(t=0\). Show
(by substitution into the Schrödinger equation is fine) that this
function evolves as

\[
  \varphi(\br,t)=\frac{1}{\left(\pi R_t^{2}\right)^{3/4}}\exp\left[-\frac{\br^2\left(1+i t/m R_0^2)\right)}{2R_t^2}\right],
 \label{Gaussian}
\]

where

\[
R_t^2=R_0^2+\left(\frac{ t}{mR_0}\right)^2.
\]
\end{quote}

\eqref{Gaussian} illustrates a very important point about the expansion
of a gas. After a long period of expansion, the final density
distribution is a reflection of the initial \emph{momentum}
distribution. This is simply because faster moving atoms fly further, so
after time \(t\) an atom with velocity \(\mathbf{v}\) will be at
position \(\mathbf{r}=\mathbf{v}t\) from the center of the trap,
provided that this distance is large compared to \(R_{0}\), the initial
radius of the gas. The \(t\to\infty\) limit of \(\eqref{Gaussian}\)
gives

\[
        |\varphi(\br,t\to\infty)|^{2}\propto \exp\left[-\left(\frac{m R_{0}\br}{ t}\right)^{2}\right],
    \label{2nd_quant_TimeOfFlight}
\]

reflecting a Gaussian initial momentum distribution of width
\(R_0^{-1}\). Imaging the density distribution after expansion is one of
the most commonly used experimental techniques in ultracold physics, and
yields information about the momentum distribution
\(n(\bp)\equiv \adop_{\bp}\aop_{\bp}\) before expansion.

Consider the evolution of two Gaussian wavepackets with width \(R_0\) at
\(t=0\), separated by a distance \(d\gg R_0\)

\[
    \varphi_{L,R}(\br,t)=\frac{1}{\left(\pi R_t^{2}\right)^{3/4}}\exp\left[-\frac{\left(\br\pm\mathbf{d}/2\right)^2\left(1+i t/m
    R_0^2)\right)}{2R_t^2}\right],
\]

The final term of \(\eqref{dens_int}\) is then

\[
\begin{align}
    \rho_{\mathrm{int}}(\br,t)&=A(\br,t)\cos\left[\theta+\frac{\br
    \cdot\mathbf{d}}{mR_0^2R_t^2}t\right]\nonumber\\
    A(\br,t)&=\frac{2\sqrt{\bar N_L\bar N_R}}{\pi^{3/2}R_t^3}\exp\left(-\frac{\br^2+\mathbf{d}^2/4}{R_t^2}\right)
\label{int_term}
\end{align}
\]

The interference term therefore consists of regularly spaced fringes,
with a separation at long times of \(2\pi t/md\).

Now we imagine doing the same thing with two condensates of fixed
particle number, which bear no phase relation to one another. The system
is described by the product state (often called a \textbf{Fock state} in
this context)

\[
    \ket{N_L,N_R}\equiv\frac{1}{\sqrt{N_L! N_R!}}\left(\adop_L\right)^{N_L}\left(\adop_R\right)^{N_R}\ket{\text{VAC}}.
\]

Computing the density in the same way yields

\[
    \rho_1(\br,t)=N_L|\varphi_L(\br,t)|^2+N_R|\varphi_R(\br,t)|^2,
\label{dens_fock}  
\]

which differs from the previous result by the absence of the
interference term.

This is not the end of the story, however. When we look at an absorption
image of the gas, we are not looking at an \emph{expectation value} of
\(\rho(\br)\) but rather the measured value of some observable(s)
\(\rho(\br)\). The expectation value just tells us the result we would
expect to get if we repeated the same experiment many times and averaged
the result. We get more information by thinking about the correlation
function of the density at two different points.

An application of our result \(\eqref{more_rho2result}\) for the density
correlations gives

\[
\begin{align}
    \rho_2(\br,\br')&=\rho_1(\br)\rho_1(\br')
    +N_LN_{R}\varphi_L^*(\br)\varphi_R^*(\br')\varphi_L(\br')\varphi_R(\br) \nonumber\\
    &\qquad+N_{L}N_R\varphi_R^*(\br)\varphi_L^*(\br')
    \varphi_R(\br')\varphi_L(\br).
  \label{dens_corr}
\end{align}
\]

We see that the second line contains interference fringes, with the same
spacing as before. The correlation function gives the relative
probability of finding an atom at \(\br'\) if there is one at \(\br\).
We conclude that in each measurement of the density, fringes are present
but with a phase that varies between measurements, even if the samples
are identically prepared.

\{\{\textless{} figure src=``AndrewsFringes.png'' title=``Fringes
observed by interfering two Bose condensates (Source:
\href{https://science.sciencemag.org/content/275/5300/637}{Andrews
\emph{et al.}}).'' numbered=``true'' lightbox=``true'' \textgreater\}\}

The rather surprising implication is that predictions for measured
quantities for a system in a Fock state are the same as in a relative
phase state, but with a subsequent averaging over the phase.

\begin{quote}
Prove this by showing that the density matrix

\[
\rho=\int_0^{2\pi}\frac{d\theta}{2\pi}\ket{\bar N_L,\bar N_R}_\theta\bra{\bar N_R,\bar N_L}_\theta
\]

coincides with that of a mixture of Fock states with binomial
distribution of atoms into states \(\varphi_{L}\), \(\varphi_{r}\). At
large \(N\) this distribution becomes sharply peaked at occupations
\(\bar N_L\), \(\bar N_R\).
\end{quote}

The interference of two independent condensates was observed in 1997 in
\{\% cite Andrews1997 \%\}. The related question of whether two
independent light sources give rise to interference was discussed much
earlier in \{\% cite Magyar1963 \%\}. The occurrence of interference
fringes in a correlation function does not depend upon Bose
condensation, although the phenomenon is very striking in this case
because the fluctuations are parametrically as large as the
\(\rho_1(\br)\rho_1(\br')\) term in \(\eqref{dens_corr}\). The general
phenomenon is known as the \textbf{Hanbury Brown and Twiss effect},
which is the work of two people (not three): Robert Hanbury Brown and
Richard Q. Twiss. For the history and early applications of this effect,
see \{\%cite Baym:1998aa \%\} and \{\% cite kleppner2008 \%\}.

\hypertarget{hartreefock-theory}{%
\subsection{Hartree--Fock Theory}\label{hartreefock-theory}}

We now apply these ideas to the approximate calculation of the energy of
an interacting many body system.

\hypertarget{the-hartree-and-fock-potentials}{%
\subsubsection{The Hartree and Fock
Potentials}\label{the-hartree-and-fock-potentials}}

Recall from last time that a two body interaction has the form

\[
\hat H_\text{int.} = \sum_{j<k} U(\br_j-\br_k)=\frac{1}{2}\int d\br_1 d\br_2\, U(\br_1-\br_2)\pdop(\br_1)\pdop(\br_2)\pop(\br_2)\pop(\br_1).
\]

Since

\[
\sum_{j<k} U(\br_j-\br_k) = \frac{1}{2}\int \sum_{j\neq k}\delta(\br_1-\br_j)\delta(\br_2-\br_k)U(\br_1-\br_2) d\br_1 d\br_2,
\]

we can immediately write down the expectation value of the interaction
energy in a product state

\[
\begin{align}
    \label{2nd_quant_HartreeFock}
    \langle \hat V\rangle &= \overbrace{\frac{1}{2}\int d\br\, d\br'\, \rho_1(\br) U(\br-\br')\rho_1(\br')}^{\equiv E_\text{Hartree}} \nonumber\\
    &\qquad\overbrace{\pm \frac{1}{2}\int d\br\, d\br'\,  U(\br-\br')g(\br,\br')g(\br',\br)}^{\equiv E_\text{Fock}}.
\end{align}
\]

The two terms are known as the \textbf{Hartree} and \textbf{Fock} (or
\textbf{exchange}) contributions, respectively. This expression lies at
the core of the variational \textbf{Hartree--Fock method} for many body
systems, which approximates the ground state by a product state. The
Hartree term looks completely reasonable, while the Fock potential
doesn't look like a potential at all, and reflects the non-classical
correlations.

\hypertarget{hartreefock-for-the-electron-gas}{%
\subsubsection{Hartree--Fock for the Electron
Gas}\label{hartreefock-for-the-electron-gas}}

How does the Hartree--Fock picture change when we have spin? Let's
consider a system of spin-1/2 fermions. We can describe such a system in
terms of field operators \(\pop_\sigma(\br)\), \(\pdop_\sigma(\br')\)
satisfying the canonical anticommutation relations

\[
\begin{gather}
    \left\{\pop_{\sigma_1}(\br_1),\pdop_{\sigma_2}(\br_2)\right\}=\delta_{\sigma_1\sigma_2}\delta(\br_1-\br_2)\nonumber\\
    \left\{\pop_{\sigma_1}(\br_1),\pop_{\sigma_2}(\br_2)\right\}=\left\{\pdop_{\sigma_1}(\br_1),\pdop_{\sigma_2}(\br_2)\right\}=0.
    \label{2nd_quant_PositionRelationsAnti}
\end{gather}
\]

The density matrix is a matrix in spin space as well as real space

\[
g_{\sigma_1\sigma_2}(\br_1,\br_2) = \braket{\Psi}{\pdop_{\sigma_1}(\br_1)\pop_{\sigma_2}(\br_2)}{\Psi}.
\]

From \(g_{\sigma_1\sigma_2}(\br_1,\br_2)\) we can extract the spin
density as well the density

\[
\mathbf{\rho}(\br) = \tr\left[g(\br,\br)\right],\quad \mathbf{s}(\br) = \frac{1}{2}\tr\left[\boldsymbol{\sigma}g(\br,\br)\right].
\]

(I've dropped the subscript from \(\rho_1(\br)\) here.) An
spin-independent interaction potential is described by a Hamiltonian of
the form

\[
\hat H_\text{int.} = \frac{1}{2}\sum_{\sigma_1,\sigma_2}\int d\br_1 d\br_2\, U(\br_1-\br_2)\pdop_{\sigma_1}(\br_1)\pdop_{\sigma_2}(\br_2)\pop_{\sigma_2}(\br_2)\pop_{\sigma_1}(\br_1).
\]

The Hartree--Fock energy is then

\[
\begin{align}
    \langle \hat H_\text{int.}\rangle &=\frac{1}{2}\int d\br\, d\br'\, \rho(\br) U(\br-\br')\rho(\br')\nonumber\\
    &- \frac{1}{2}\int d\br\, d\br'\,  U(\br-\br')\tr\left[g(\br,\br')g(\br',\br)\right].
  \label{2nd_quant_HFSpin}
\end{align}
\]

The Fock term can be rewritten in a more useful way using the identity

\[
\delta_{ab}\delta_{cd} = \frac{1}{2}\left[\boldsymbol{\sigma}_{a c}\cdot \boldsymbol{\sigma}_{d b} + \delta_{ac}\delta_{db}\right].
\label{pauli_ident}
\]

\begin{quote}
One way to understand \(\eqref{pauli_ident}\) is to to think of the two
Pauli matrices as acting on two spin 1/2s (as we did in {[}Lecture
4{]}(\{\{\textless{} ref ``spin-models'' \textgreater\}\})), in which
case we can work in the basis \(\ket{s_1}\ket{s_2}\) and we have \[
\boldsymbol{\sigma}_1\cdot \boldsymbol{\sigma}_2+ \mathbb{1}_1\mathbb{1}_2 = 2\begin{pmatrix}
1 & 0 & 0 & 0 \\
0 & 0 & 1 & 0 \\
0 & 1 & 0 & 0 \\
0 & 0 & 0 & 1 
\end{pmatrix}.
\] The matrix elements of the right hand side are then
\(2\delta_{s_1,s_2'}\delta_{s_2,s_1'}\).
\end{quote}

which gives

\[
\begin{align}
E_{\text{Fock}} &=-\frac{1}{4} \int d\br\, d\br'\,  U(\br-\br')\tr\left[g(\br,\br')\right]\tr\left[g(\br',\br)\right]\nonumber\\&-\frac{1}{4}\int d\br\, d\br'\,  U(\br-\br')\tr\left[\boldsymbol{\sigma}g(\br,\br')\right]\cdot\tr\left[\boldsymbol{\sigma}g(\br',\br)\right].
\end{align}
\]

Suppose we had a \(\delta\)-function interaction
\(U(\br)=V_0 \delta(\br)\). Then the Fock energy can be written

\[
\begin{align}
E_{\text{Fock}} =-\frac{V_0}{4} \int d\br\, \rho(\br)^2-V_0\int d\br\, \mathbf{s}(\br)\cdot\mathbf{s}(\br)
\end{align}
\]

The second term favours ferromagnetism for repulsive interactions. The
physical origin is the same as the Hund's rule coupling in atoms:
fermions in different spin states can sit at the same spatial location,
while those in the same spin state must be in different locations. For
repulsive interactions occupying the same spin state is energetically
favourable.

\begin{quote}
This is most succintly put by the formula

\[
\rho_2(\br,\br) = \frac{1}{2}\rho(\br)^2 - 2\mathbf{s}(\br)\cdot\mathbf{s}(\br)
\]
\end{quote}

The Hartree--Fock energy forms the basis of a variational method using
product states as variational wavefunctions. For a Hamiltonian with
translational invariance, like

\[
H = \int d\br \frac{1}{2m}\nabla\pdop\cdot\nabla\pop + \frac{1}{2}\int d\br d\br' U(\br-\br')\pdop(\br)\pdop(\br')\pop(\br')\pop(\br),
\label{more_H2nd}
\]

this is not too bad, as we are guaranteed to be working with plane wave
single particle states. Then the only variational parameters are the
occupancies of these states: we'll meet an example in the next section.
If translational symmetry is broken by introducing a potential
\(U(\br)\) into the single particle part of \(\eqref{more_H2nd}\), say,
the situation is more complicated. We may be tempted to use single
particle states that are the eigenstates of the single particle part of
the Hamiltonian, but there is no real justification for this, and one
has to allow the states, as well as the occupancies, to vary.

\hypertarget{stoner-criterion-for-ferromagnetism}{%
\subsubsection{Stoner Criterion for
Ferromagnetism}\label{stoner-criterion-for-ferromagnetism}}

Let us try to put a bit more flesh on the idea that repulsive
interactions favour ferromagnetism in fermionic systems. We will
continue to use the model interaction \(U(\br)=V_0\delta(\br)\) that we
introduced in the previous section. Of course, this isn't a realistic
interaction between electrons in a metal, say, but as we'll see in
{[}Lecture 11{]}(\{\{\textless{} ref ``jellium'' \textgreater\}\}), the
long-ranged Coulomb interaction is screened and becomes finite-ranged.
Thus our model is not a bad approximation to the \emph{effective}
interaction between electrons in a metal.

Polarizing the spins in a Fermi gas is not without cost (otherwise
everything would be ferromagnetic!): there is a price to pay in
increased kinetic energy. To understand why this is so, consider the
ground state kinetic energy of \(N\) (spinless) fermions in three
dimensions, obtained from

\[
\begin{align}
N = \sum_{|\bk|<k_\text{F}} 1 &\longrightarrow L^3 \int_{|\bk|<k_\text{F}} \frac{d\bk}{(2\pi)^3} = \frac{k_\text{F}^3 L^3}{6\pi^2} \nonumber\\
E_\text{kin} = \sum_{|\bk|<k_\text{F}} \frac{\bk^2}{2m} &\longrightarrow L^3 \int_{|\bk|<k_\text{F}} \frac{d\bk}{(2\pi)^3} \frac{\bk^2}{2m}\\
 &= \frac{k_\text{F}^5 L^3}{20\pi^2 m} = L^3 \frac{3}{10m}(6\pi^2)^{2/3} n^{5/3},\nonumber
\end{align}
\]

(The assumption of a quadratic dispersion is not important here. More
generally, we fill a band structure.) where \(n = \frac{N}{L^3}\) is the
mean density. Assuming a system of fermions with spin is now polarized
in the \(z\)-direction, we have differing densities
\(n_{\uparrow,\downarrow}\) of spin up and spin down fermions. Their
total energy is

\[
E_\text{kin}(n_\uparrow,n_\downarrow) = \frac{cL^3}{m}\left(n_\uparrow^{5/3}+n_\downarrow^{5/3}\right),
\]

where \(c=\frac{3}{10}(6\pi^2)^{2/3}\). In terms of the overall density
\(n=n_\uparrow+n_\downarrow\) and density of spin
\(\bar s = \left(n_\uparrow-n_\downarrow\right)/2\), we have

\[
E_\text{kin}(n, \bar s) = \frac{cL^3}{m}\left(\left[n/2+\bar s\right]^{5/3}+\left[n/2-\bar s\right]^{5/3}\right).
\]

Alternatively, write this in terms of the \textbf{polarization}
\(P \equiv \frac{n_\uparrow-n_\downarrow}{n}\) that varies in the range
\([-1,1]\) as

\[
E_\text{kin}(P) = \frac{E_\text{K}}{2}\left[(1+P)^{5/3}+(1-P)^{5/3}\right].
\]

We see that, on account of the convexity of \(x^{5/3}\),
\(E^{(0)}_\text{kin}(n, \bar s)\) is minimized for \(s=0\).

Let's compare this with the effect of interactions. In the short-ranged
model introduced above, the total Hartree--Fock energy is

\[
E_\text{HF}(n,\bar s) = \frac{V_0L^3}{2} n^2 - \frac{V_0L^3}{2} \left(n_\uparrow^2+n_\downarrow^2\right) =  \frac{V_0L^3}{2} \left(\frac{1}{2}n^2 - 2\bar s^2\right).
\]

We write this in terms of the polarization as

\[
E_\text{HF}(P) = \frac{E_V}{2}(1-P^2).
\]

\begin{quote}
Minimize the total energy \(E(P) = E_\text{kin}(P) + E_\text{HF}(P)\) to
show

\begin{enumerate}
\def\labelenumi{\arabic{enumi}.}
\tightlist
\item
  For \(E_V/E_K<10/9\) the ground state is non-magnetic.
\item
  As \(E_V/E_K\) increases past \(10/9\) the magnetization begins to
  increase.
\item
  At \(E_V/E_K>\frac{5}{6}2^{2/3}\) is the ground state is fully
  polarized.
\end{enumerate}
\end{quote}

One shouldn't take these numerical values too seriously given the
simplicity of the model, but they illustrate the physical principles at
work behind the appearance of ferromagnetism in metals.

\hypertarget{excited-state-energies}{%
\subsubsection{Excited State Energies}\label{excited-state-energies}}

So far we have discussed properties of the ground state only. We can,
however, evaluate the Hartree--Fock energy in a product state describing
an excited state of a noninteracting system. This gives the first order
perturbation theory correction to the excited state energy. For
concreteness we will stick with fermions for now, though the method is
general.

If we work in a translationally invariant system, the appropriate single
particle states are plane waves. We write the field operators in the
plane wave basis as

\[
\begin{align}
    \pop(\br)\equiv\frac{1}{L^{3/2}}\sum_{\bk} \exp(i\bk\cdot\br)\aop_{\bk},\nonumber\\
  \pdop(\br)\equiv\frac{1}{L^{3/2}}\sum_{\bk} \exp(-i\bk\cdot\br)\adop_{\bk},
\end{align}
\]

and represent the interaction potential in terms of its Fourier
components

\[
U(\br-\br') = \frac{1}{L^3}\sum_\bq \tilde U(\bq) \exp(i\bq\cdot[\br-\br']).
\]

The interaction Hamiltonian for spinless particles can then be written

\[
\hat H_\text{int.}  = \frac{1}{2L^3} \sum_{\bk_1+\bk_2=\bk_3+\bk_4} \tilde U(\bk_1-\bk_4) \adop_{\bk_1}\adop_{\bk_2}\aop_{\bk_3}\aop_{\bk_4}.
\label{more_vertex}
\]

When written in this way, interaction Hamiltonians are sometimes
associated with the graphical representation below, one of the
ingredients of the \textbf{Feynman diagram} technique for performing
perturbation theory calculations in field theories. The incoming lines
(arrows in) represent particles being removed (in momentum states
\(\bk_3\) and \(\bk_4\)) and the outgoing lines represent particles
added (momenta \(\bk_1\) and \(\bk_2\)). The wiggly line represents
\(\tilde V(\bq)\). The conservation of momentum at the vertices is a
consequence of the translational invariance of the problem: the two
integrations over \(\br\) and \(\br'\) give rise to two
\(\delta\)-functions that perform this function.

\{\{\textless{} figure src=``Vertex.png'' title=``Graphical
representation of the interaction \(\eqref{more_vertex}\).''
numbered=``true'' lightbox=``true'' \textgreater\}\}

As we've already discussed, the expectation value of
\(\eqref{more_vertex}\) in a product state of momentum eigenstates gives
two terms, with different `pairings' of creation operators with
annihilation operators. We can represent these two terms graphically as
shown below.

\{\{\textless{} figure src=``HFDiag.png'' title=``Graphical
representation of the Hartree and Fock terms.'' numbered=``true''
lightbox=``true'' \textgreater\}\}

Evaluating the two contributions in terms of the occupation numbers
gives

\[
\braket{\mathbf{N}}{\hat H_\text{int.}}{\mathbf{N}} = \frac{1}{2V}\tilde U(0) \sum_{\bk_1,\bk_2} N_{\bk_1}N_{\bk_2} - \frac{1}{2V} \sum_{\bk_1,\bk_2} \tilde U(\bk_1-\bk_2) N_{\bk_1}N_{\bk_2}
\]

While the Hartree term just depends on the total number of particles,
the Fock term depends on the individual occupations. The interaction
energy to add a single particle to state \(\bk\) is

\[
\Delta U_{\bk} = \frac{\tilde U(0)}{V} \sum_{\bk'} N_{\bk'} - \frac{1}{V}\sum_{\bk'} \tilde U(\bk-\bk') N_{\bk'}
\]

\hypertarget{references}{%
\subsection{References}\label{references}}

\{\% bibliography --cited \%\}
